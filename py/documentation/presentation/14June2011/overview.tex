\documentclass{beamer}

\mode<presentation>

\usetheme{Frankfurt}%
\usecolortheme{seagull}
\usepackage{multimedia}
%\usepackage{movie15}

\logo{\includegraphics[height=.25in]{clarksonGreen}}

%\definecolor{garnet}{RGB}{179,0,91}
%\definecolor{garnet}{RGB}{108,43,53}
\definecolor{garnet}{RGB}{136,0,0}
%\definecolor{clarksonGreen}{RGB}{0,71,28}
\definecolor{clarksonGreen}{RGB}{0,52,21}
\setbeamercolor{palette primary}{fg=clarksonGreen,bg=white}
\setbeamercolor{palette secondary}{fg=clarksonGreen,bg=white}
\setbeamercolor{palette tertiary}{fg=clarksonGreen,bg=white}
\setbeamercolor{palette quaternary}{bg=clarksonGreen,fg=white}
\setbeamercolor{block title}{fg=black,bg=black!15}
\setbeamercolor{block body}{fg=black,bg=black!10}
\setbeamercolor{titlelike}{bg=clarksonGreen,fg=white} % parent=palette quaternary}



% Misc definitions that are nice to have
\newcommand{\pd}{\partial}
\newcommand{\half}{\frac{1}{2}}
\newcommand{\fullInt}{\int^\infty_{-\infty}}
\newcommand{\timeInt}{\int^T_0}
\newcommand{\partialDeriv}[1]{\frac{\partial}{\partial #1}}
\newcommand{\timeDeriv}{\frac{d}{dT}}
\newcommand{\sDeriv}{\frac{d}{ds}}
\newcommand{\lp}{\left(}
\newcommand{\rp}{\right)}
\newcommand{\tbar}{\bar{t}}
\newcommand{\that}{\hat{t}}
\newcommand{\Ahat}{{\hat{A}}}
\newcommand{\gHat}{{\hat{g}}}
\newcommand{\weight}{e^{-(\xi-\tbar/\sigma)^2}}
\newcommand{\bweight}{e^{-(\xi-b)^2}}
\newcommand{\bratio}{\frac{1}{4b}}
\newcommand{\normSquared}{\|\Ahat\|^2_w}



%\usepackage{pgfpages}
%%\pgfpagesuselayout{resize}[border shrink=5mm,landscape]
%\pgfpagesuselayout{2 on 1}[border shrink=5mm]



\begin{document}

\title{Mission Model}
\subtitle{Software Development}
\author{MADC Team}
\institute{Clarkson University}
%\date{8 July 2009}
\date{14 June 2011}

\begin{frame}
  \titlepage
  \begin{abstract}
    Borad overview of the software development of the mission
    model. The mission model is implemented in python with each agent
    represented by a separate class. Underlying supporting classes
    encapsulate the way information is exchanged and interpreted.
  \end{abstract}
\end{frame}



\begin{frame}
  \frametitle{Schedule}

  \begin{itemize}
  \item Software Development (This discussion) 
  \item Core and Implementation of the Network Model
  \item Lunch
  \item ???? - what goes here?
  \end{itemize}

\end{frame}


\begin{frame}
  \frametitle{Outline}
  \tableofcontents[pausesection,hideallsubsections]
\end{frame}



\section{Conceptual Model}


\begin{frame}
  \frametitle{Conceptual Model}

\end{frame}


\begin{frame}
  \frametitle{Computational Model}
  %\includegraphics[height=8cm]{compareNoiseNoNoise.png}
\end{frame}


\begin{frame}
  \frametitle{Why}

  This is the subject of Todd and Pat's discussion which will follow.

\end{frame}


\section{Simulation Requirements}

\begin{frame}
  \frametitle{Simulation}


\end{frame}


\begin{frame}
  \frametitle{Goals}

  %\centerline{\includegraphics[height=6.0cm]{nearEquilibrium}}


\end{frame}

\begin{frame}
  \frametitle{Communication Goals}

  %\centerline{\includegraphics[height=6.0cm]{awayEquilibrium}}


\end{frame}


\section{Implementation Details}

\begin{frame}
  \frametitle{Core}

  Todd and Pat will discuss this.

\end{frame}


\begin{frame}
  \frametitle{Encapsulating Agent Communications}


\end{frame}


\begin{frame}
  \frametitle{Data Exchange Considerations}
  

\end{frame}




%\begin{frame}
%  \frametitle{Test Cases}
%
%  $r=2$, $\alpha=0.5$.
%
%  \begin{columns}[t]
%    \column{.5\textwidth}
%    \centerline{\includegraphics[height=5.0cm]{meanError-r2-alpha0_5}}
%
%    \column{.5\textwidth}
%    \centerline{\includegraphics[height=5.0cm]{varError-r2-alpha0_5}}
%  \end{columns}
%
%\end{frame}



%  \begin{columns}[t]
%    
%    \column{.5\textwidth}
%    \begin{block}{Barney?}
%      Is it okay to trust your kids with Barney?
%    \end{block}
%    \pause
%  
%    \column{.5\textwidth}
%    \begin{block}{No, not Barney!}
%      Probably not.
%    \end{block}
%
%  \end{columns}
%
%
%\end{frame}





\begin{frame}
  \frametitle{Acknowledgements}
  Is this redundant? If not what should be here?
\end{frame}



\end{document}
